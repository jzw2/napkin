

 
\documentclass{article}
\usepackage{amsmath}
\usepackage{amssymb}

\begin{document}

\textbf{Prove Lagrange’s theorem for orders in the special case that G is a finite
abelian group.}


Let $G = \{g_1, g_2, g_3, \dots, g_n\}$ and 
Let $g \in G$. Let $h = g_1g_2g_3\dots g_n $  The map $x \mapsto gx $ is a bijection,
so $h = g g_1 g g_2 g g_3 \dots g g_n$ for some permutation of $g_i$.  However,  because G is abelian
h is the same no matter the permutation.  Then, we can simplify this to
$h = g^n h$ therefore $g^n$ is the identity.

\textbf{Let p be a prime. Show that the only group of order p is Z/pZ.}


Let $G$ be a group with order $p$. Let $0$ be the identity element. $p$ is prime, so $p \ge 2$, which means there must
be at least one other element $g$ which is not the identity element. Let $H$ be the subgroup
generated by $g$. If $|H| = |G|$, then we are done through the map $n \mapsto g^n$. 

Assume then that $|H| \ne |G|$. $|H|$ has to be smaller than $|G|$, because otherwise $G$ is not closed.  
By lagrange's theorem,  $g^{|H|} = 0$, and $g^{|G|} = 0$, so $g ^{k |H| \mod |G|} = 0$, for $k \in \mathbb{N}$


\((Z / pZ)^{\times}\) is a group with size \(p - 1\), so therefore by Lagrange's theorem, for any
\(x \in (Z / pZ)^{\times}\),

\begin{equation}
\label{eq:1}
x^{p-1} = 1 \pmod p
\end{equation}

Equation \ref{eq:1} is fermat's little theorem.
Since we know $|G|$ is prime, by Fermat's Little theorem, $|H|^{|G| - 1} \mod |G| = 1$,

so $g = 0$, but we said that $g$ was not the identity, so $|H| = |G|$, and they 
are isomorphic.


\textbf{Let $p$ be a prime and $F_1 = F_2 = 1, F_{n+2} = F_{n+1} + F_n$
 be the Fibonacci sequence. Show that $F_{2p(p^2-1)}$ is divisible by $p$}.


We can turn the fibonacci sequence into a matrix using


\[
g = 
\begin{pmatrix}
1 & 1 \\
1 & 0 \\
\end{pmatrix} 
\]
because 
\[
\begin{pmatrix}
1 & 1 \\
1 & 0 \\
\end{pmatrix}^n = 
\begin{pmatrix}
F_{n+1} & F_{n} \\
F_n & F_{n -1 }\\
\end{pmatrix}
\]

This is proved using induction.  The base case is $n = 1$ and is true, then 

\[
g^{n + 1} = g g^n = \begin{pmatrix}
1 & 1 \\
1 & 0 \\
\end{pmatrix}
\begin{pmatrix}
F_{n + 1} & F_{n} \\
F_n & F_{n - 1} \\
\end{pmatrix} = 
\begin{pmatrix}
F_{n + 2}  & F_{n + 1} \\
F_{n + 1} & F_n \\
\end{pmatrix}
\]

If the field of the matrix is $\mathbb{Z} / p \mathbb{Z}$, and we prove
that $g^n = I$, where $I$ is the identity matrix, then we will have shown that
$F_n = 0 \mod p$.


Observe that the determinant of $g$  is $-1$. Note that the set of all 2 by 2 matrices 
mod p
with determinant $\pm 1$ forms a group. It has an identity element,
matrix multiplication is associative, and the inverse of each matrix 
also has the determinant $\pm 1$.  

Let this group be $G$.  Then all elements of this group are forms of $ad - bc = \pm 1$,
$a, b, c, d $ greater than equal $0$ and  less than $p$. If we can show that 
$|G| = 2p(p^2 - 1)$, then by Lagrange's theorem, $g^{|G|} = I$,completing the proof.


For now consider forms of $ad - bc = 1$
For any value $ad$, there exists a unique value that $bc$ must be to
satisfy the equation.

Split this into cases where $ad = 1$ and $ad \ne 1$

\textbf{case 1}
If $ad = 1 \mod p$, then both $a$ and $d$ canot be $0$, and if $a$ is non zero
then there is a unique vaule that $d$ must be, so there are $p - 1$ pairs of $a, d$
that satisfy $ad = 1 \mod p$.  Then $bc = 0 \mod p$, so $b$ or $c$ must be $0$, so
there are $2p - 1$ pairs of $b, c$, that satisfy this.  Therefore, there are
$(p - 1)(2p - 1)$ total.

\textbf{case 2:}
If $ad \ne 1$, then of the $p^2$ total pairs of $a, d$, we subtract those that have
$ad = 1$, leaving us with $p^2 - p + 1$ pairs.  By the same reason that 
there are $p - 1$ pairs that satisfy $ad = 1$, there are $p -1$ pairs of $b, c$ 
that wil satisfy $bc = 1 - ad$, leaving $(p^2 - p + 1)(p - 1)$ total.

Combining the cases, we get $(p - 1)(p^2 + p)$ matrices that have determinant $1$.
By a similar proof, we can show there are $(p - 1)(p^2 + p)$ matrices that have
determinant $-1$.  In total there are $2(p-1)(p^2 + p) = 2p(p^2 - 1)$, so 
$|G| = 2p(p^2 - 1)$, which completes the proof.


\end{document}

