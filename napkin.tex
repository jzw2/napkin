% Created 2022-07-19 Tue 18:10
% Intended LaTeX compiler: pdflatex
\documentclass[11pt]{article}
\usepackage[utf8]{inputenc}
\usepackage[T1]{fontenc}
\usepackage{graphicx}
\usepackage{longtable}
\usepackage{wrapfig}
\usepackage{rotating}
\usepackage[normalem]{ulem}
\usepackage{amsmath}
\usepackage{amssymb}
\usepackage{capt-of}
\usepackage{hyperref}
\author{John Wang}
\date{\today}
\title{Napkin}
\hypersetup{
 pdfauthor={John Wang},
 pdftitle={Napkin},
 pdfkeywords={},
 pdfsubject={},
 pdfcreator={Emacs 28.1 (Org mode 9.6)}, 
 pdflang={English}}
\begin{document}

\maketitle
\tableofcontents


\section{Groups}
\label{sec:org61a8ec1}
\subsection{B}
\label{sec:org16ed8ec}
Prove Lagrange’s theorem for orders in the special case that \(G\) is a finite abelian group.


\subsubsection{Proof}
\label{sec:org8dc98ef}
Let \(G = \{g_1, g_2, g_3, \dots, g_n\}\) and
Let \(g \in G\). Let \(h = g_1g_2g_3\dots g_n\)  The map \(x \mapsto gx\) is a bijection,
so \(h = g g_1 g g_2 g g_3 \dots g g_n\) for some permutation of \(g_i\).  However,  because G is abelian
h is the same no matter the permutation.  Then, we can simplify this to
\(h = g^n h\) therefore \(g^n\) is the identity.

\subsection{D}
\label{sec:orgc55dca2}
Let p be a prime. Show that the only group of order p is \(\mathbb{Z}/p \mathbb{Z}\).


\subsubsection{Proof}
\label{sec:org079afa4}
Let \(G\) be a group with order \(p\). Let \(0\) be the identity element. \(p\) is prime, so \(p \ge 2\), which means there must
be at least one other element \(g\) which is not the identity element. Let \(H\) be the subgroup
generated by \(g\). If \(|H| = |G|\), then we are done through the map \(n \mapsto g^n\).

Assume then that \(|H| \ne |G|\). \(|H|\) has to be smaller than \(|G|\), because otherwise \(G\) is not closed.
By lagrange's theorem,  \(g^{|H|} = 0\), and \(g^{|G|} = 0\), so \(g ^{k |H| \mod |G|} = 0\), for \(k \in \mathbb{N}\)


\((Z / pZ)^{\times}\) is a group with size \(p - 1\), so therefore by Lagrange's theorem, for any
\(x \in (Z / pZ)^{\times}\),

\begin{equation}
\label{fermatlittle}
x^{p-1} = 1 \pmod p
\end{equation}

Equation \ref{fermatlittle} is fermat's little theorem.
Since we know \(|G|\) is prime, by Fermat's Little theorem, \(|H|^{|G| - 1} \mod |G| = 1\),

so \(g = 0\), but we said that \(g\) was not the identity, so \(|H| = |G|\), and they
are isomorphic.


\subsection{H}
\label{sec:orgeabbfef}

Let \(p\) be a prime and \(F_1 = F_2 = 1, F_{n+2} = F_{n+1} + F_n\)
 be the Fibonacci sequence. Show that \(F_{2p(p^2-1)}\) is divisible by \(p\).

\subsubsection{Proof}
\label{sec:orga45c51e}
We can turn the fibonacci sequence into a matrix using


\begin{equation}
\label{eq:2}
g =
\begin{pmatrix}
1 & 1 \\
1 & 0 \\
\end{pmatrix}
\end{equation}


because
\begin{equation}
\label{eq:3}
\begin{pmatrix}
1 & 1 \\
1 & 0 \\
\end{pmatrix}^n =
\begin{pmatrix}
F_{n+1} & F_{n} \\
F_n & F_{n -1 }\\
\end{pmatrix}
\end{equation}
This is proved using induction.  The base case is \(n = 1\) and is true, then


\begin{equation}
\label{eq:4}
g^{n + 1} = g g^n = \begin{pmatrix}
1 & 1 \\
1 & 0 \\
\end{pmatrix}
\begin{pmatrix}
F_{n + 1} & F_{n} \\
F_n & F_{n - 1} \\
\end{pmatrix} =
\begin{pmatrix}
F_{n + 2}  & F_{n + 1} \\
F_{n + 1} & F_n \\
\end{pmatrix}
\end{equation}

If the field of the matrix is \(\mathbb{Z} / p \mathbb{Z}\), and we prove
that \(g^n = I\), where \(I\) is the identity matrix, then we will have shown that
\(F_n = 0 \mod p\).


Observe that the determinant of \(g\)  is \(-1\). Note that the set of all 2 by 2 matrices
mod p
with determinant \(\pm 1\) forms a group. It has an identity element,
matrix multiplication is associative, and the inverse of each matrix
also has the determinant \(\pm 1\).

Let this group be \(G\).  Then all elements of this group are forms of \(ad - bc = \pm 1\),
\(a, b, c, d\) greater than equal \(0\) and  less than \(p\). If we can show that
\(|G| = 2p(p^2 - 1)\), then by Lagrange's theorem, \(g^{|G|} = I\),completing the proof.


For now consider forms of \(ad - bc = 1\)
For any value \(ad\), there exists a unique value that \(bc\) must be to
satisfy the equation.

Split this into cases where \(ad = 1\) and \(ad \ne 1\)

\textbf{case 1}
If \(ad = 1 \mod p\), then both \(a\) and \(d\) canot be \(0\), and if \(a\) is non zero
then there is a unique vaule that \(d\) must be, so there are \(p - 1\) pairs of \(a, d\)
that satisfy \(ad = 1 \mod p\).  Then \(bc = 0 \mod p\), so \(b\) or \(c\) must be \(0\), so
there are \(2p - 1\) pairs of \(b, c\), that satisfy this.  Therefore, there are
\((p - 1)(2p - 1)\) total.

\textbf{case 2:}
If \(ad \ne 1\), then of the \(p^2\) total pairs of \(a, d\), we subtract those that have
\(ad = 1\), leaving us with \(p^2 - p + 1\) pairs.  By the same reason that
there are \(p - 1\) pairs that satisfy \(ad = 1\), there are \(p -1\) pairs of \(b, c\)
that wil satisfy \(bc = 1 - ad\), leaving \((p^2 - p + 1)(p - 1)\) total.

Combining the cases, we get \((p - 1)(p^2 + p)\) matrices that have determinant \(1\).
By a similar proof, we can show there are \((p - 1)(p^2 + p)\) matrices that have
determinant \(-1\).  In total there are \(2(p-1)(p^2 + p) = 2p(p^2 - 1)\), so
\(|G| = 2p(p^2 - 1)\), which completes the proof.

\section{Metric Spaces}
\label{sec:orgcea62a3}

\subsection{Exercise 2.3.4}
\label{sec:org1364005}
Show that \(\epsilon -\delta\) continuity implies sequential continuitay at
each point.


\subsubsection{Proof}
\label{sec:org903c424}
Let \(p\) be the continuous point for \(f\).

It is needs to be shown that  \(x_1, x_2, \ldots\) is a sequene in \(M\) is
coverging to \(p\), then the sequence, \(f(x_1), f(x_2), f(x_3), \ldots\)
covergences to \(f(p)\)


To show convergence for \(f(x_1), f(x_2), f(x_3), \ldots\)
covergences to \(f(p)\), given any \(\varepsilon\), it needs to be shown that there
exists a positive integer \(A\), such that for any \(a > A\), \(d(f(x_a), f(p)) < \varepsilon\).

Since \(\varepsilon -\delta\) continuity is assumed, that means that there is a
\(\delta\) such that

\begin{equation}
\label{eq:5}
d(x, p) < \delta \Rightarrow d(f(x), f(p)) < \varepsilon
\end{equation}

Because \(x_1, x_2, x_3, \ldots\) converges, it menas that there is an
integer \(A\) such that for any \(a > A\), \(d(x_{a}, p) < \delta\), but by equation
\ref{eq:5}, this means that \(d(f(x_{a}), f(p)) < \varepsilon\), and so this
concludes the proof.




\section{Homomorphism and Quotient Groups}
\label{sec:orgdcd2c9d}
\subsection{A}
\label{sec:orgd0e42fe}

\begin{quote}
Determine all groups \(G\) for which the map \(\phi : G \rightarrow G\) defined by
\begin{equation}
\label{eq:1}
\phi(g) = g^{2}
\end{equation}
is a homomorphism.
\end{quote}
\subsubsection{Proof}
\label{sec:org2d02dc0}
By definiton of homomorphism, for any \(g_1, g_2\), \(\phi(g_1 g_2) =
\phi(g_{1}) \phi(g_{2})\), so \((g_1 g_{2})^{2}  =
g_{1}^{2} g_{2}^{2}\), so \(g_1 g_2 g_1 g_2 = g_1 g_1 g_2 g_2\) so \(g_2 g_1 =
g_1 g_2\), Therefore, these groups are abelian.
\subsection{C}
\label{sec:org0b040b7}


Does \(S_4\) have a normal subgroup of order 3?
\subsubsection{Answer}
\label{sec:org8452dfc}

Yes, take the element that maps \((1,2,3,4\) to \((1, 3,4,2)\). Then the
subgroup \(H\) generated by this element consists of \((1,2,3,4), (1,3,4,2), (1,4,2,3)\)


Let \(g\) map to a permutation \(x_1, x_2, x_3, x_4\), we must show that
\(ghg^{-1} \in H\). Enumerate \(h\), if \(h\) is identity, it is trivial. If \(h
= (1,3,4,2)\), then \(ghg^{-1} = (1,4,2,3)\)

(1,2,3,4)
(4,3,2,1)


gh = (4,2,1,3)
ghg\textsuperscript{-1} = (3,1,2,4)



ghg\textsuperscript{-1}
\end{document}
